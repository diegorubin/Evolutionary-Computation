\documentclass[12pt]{article}

\usepackage{sbc-template}

\usepackage{graphicx,url}

%\usepackage[brazil]{babel}   
\usepackage[latin1]{inputenc}  

     
\sloppy

\title{Buscando Valor M�ximo de Uma Fun��o Utilizando \\Algoritimos Gen�ticos}

\author{Diego Rubin\inst{1}}


\address{Departamento de Matem�tica Aplicada e Computa��o \\
  - Universidade Estadual Paulista "Julio de Mesquita Filho"
  (UNESP)\\
  Rio Claro -- SP -- Brazil
  \email{\{rubin.diego\}@gmail.com}
}

\begin{document} 

\maketitle

\begin{resumo} 
  Este artigo tem como objetivo mostrar os resultados obtidos para um
  problema de maximiza��o de uma fun��o utilizando uma t�cnica de 
  programa��o chamado de Algoritimos Gen�ticos.
\end{resumo}


\section{Introdu��o}

O principal prop�sito de um programa escrito utilizando Algoritimo Gen�tico
� encontrar um valor, m�ximo ou m�nimo, para uma fun��o, que damos o
nome de fun��o objetivo.
Ser� mostrado neste artigo um problema de maximiza��o e como a t�cnica
do foi empregada para encontrar os valores das variaveis que chegam
a um bom valor para a fun��o objetivo.

\section{O Problema} \label{sec:firstpage}

O Problema que iremos abordar � o seguinte:

A Capit�o Caverna S.A., localizada em Pedra Lascada, aluga 3 tipos de barcos para
passeios mar�timos: jangadas, supercanoas e arcas com cabine. A companhia fornece
juntamente com o barco um capit�o para naveg�-lo e uma tripula��o que varia de acordo com a
embarca��o: uma para jangadas, duas para supercanoas e tr�s para arcas. A companhia tem 4
jangadas, 8 supercanoas e 3 arcas e em seu corpo de funcion�rios: 10 capit�es e 18 tripulantes.
O aluguel � por di�rias e a Capit�o Caverna lucra R\$ 50 por jangada, R\$ 70 por supercanoa e R\$
100 por arca. Fa�a um modelo de algoritmo gen�tico que determine o esquema de aluguel que
maximiza o lucro.


\section{CD-ROMs and Printed Proceedings}

In some conferences, the papers are published on CD-ROM while only the
abstract is published in the printed Proceedings. In this case, authors are
invited to prepare two final versions of the paper. One, complete, to be
published on the CD and the other, containing only the first page, with
abstract and ``resumo'' (for papers in Portuguese).

\section{Sections and Paragraphs}

Section titles must be in boldface, 13pt, flush left. There should be an extra
12 pt of space before each title. Section numbering is optional. The first
paragraph of each section should not be indented, while the first lines of
subsequent paragraphs should be indented by 1.27 cm.

\subsection{Subsections}

The subsection titles must be in boldface, 12pt, flush left.

\section{Figures and Captions}\label{sec:figs}


Figure and table captions should be centered if less than one line
(Figure~\ref{fig:exampleFig1}), otherwise justified and indented by 0.8cm on
both margins, as shown in Figure~\ref{fig:exampleFig2}. The caption font must
be Helvetica, 10 point, boldface, with 6 points of space before and after each
caption.

\begin{figure}[ht]
\centering
\includegraphics[width=.5\textwidth]{fig1.jpg}
\caption{A typical figure}
\label{fig:exampleFig1}
\end{figure}

\begin{figure}[ht]
\centering
\includegraphics[width=.3\textwidth]{fig2.jpg}
\caption{This figure is an example of a figure caption taking more than one
  line and justified considering margins mentioned in Section~\ref{sec:figs}.}
\label{fig:exampleFig2}
\end{figure}

In tables, try to avoid the use of colored or shaded backgrounds, and avoid
thick, doubled, or unnecessary framing lines. When reporting empirical data,
do not use more decimal digits than warranted by their precision and
reproducibility. Table caption must be placed before the table (see Table 1)
and the font used must also be Helvetica, 10 point, boldface, with 6 points of
space before and after each caption.

\begin{table}[ht]
\centering
\caption{Variables to be considered on the evaluation of interaction
  techniques}
\label{tab:exTable1}
\includegraphics[width=.7\textwidth]{table.jpg}
\end{table}

\section{Images}

All images and illustrations should be in black-and-white, or gray tones,
excepting for the papers that will be electronically available (on CD-ROMs,
internet, etc.). The image resolution on paper should be about 600 dpi for
black-and-white images, and 150-300 dpi for grayscale images.  Do not include
images with excessive resolution, as they may take hours to print, without any
visible difference in the result. 

\section{References}

Bibliographic references must be unambiguous and uniform.  We recommend giving
the author names references in brackets, e.g. \cite{knuth:84},
\cite{boulic:91}, and \cite{smith:99}.

The references must be listed using 12 point font size, with 6 points of space
before each reference. The first line of each reference should not be
indented, while the subsequent should be indented by 0.5 cm.

\bibliographystyle{sbc}
\bibliography{sbc-template}

\end{document}
