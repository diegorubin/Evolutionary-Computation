\documentclass[12pt]{article}

\usepackage{sbc-template}

\usepackage{graphicx,url}

\usepackage[brazil]{babel}   
\usepackage[utf8]{inputenc}  

     
\sloppy

\title{Resolução de um Problema NP-Completo utilizando ACO}

\author{Diego Rubin\inst{1}}


\address{Departamento de Matemática Aplicada e Computação \\
  - Universidade Estadual Paulista "Julio de Mesquita Filho"
  (UNESP)\\
  Rio Claro -- SP -- Brazil
  \email{rubin.diego@gmail.com}
}

\begin{document} 

\maketitle

\begin{abstract}
\end{abstract}

\begin{resumo} 
 Este artigo tem como objetivo mostrar os
 resultados obtidos na utilização do algoritmo
 Ant System para a resolução de um problema
 de otimização.
\end{resumo}


\section{Introdução}

Problemas de otimização são frequentes no nosso coditiano e exitems 
diversas formas de encontrarmos soluções para os mesmos. Neste artigo
será apresentado um exemplo de solução para um problema de otimização
utilizando um algoritmo chamado Ant System.
O algoritmo foi implementado em uma linguagem chamada Python e o 
código gerado pelo ser encontrado em http://github.com/diegorubin/Evolutionary-Computation/tree/master/final/src.

\section{O Problema} \label{sec:firstpage}

O Problema que iremos abordar é o seguinte:

Você possui uma empresa reconhecida na área de otimizacao e algoritmos. 
Um dia e chamado por telefone para uma conversa urgente com um grupo 
de pessoas que está trabalhando em um grande projeto: “Salvar o mundo”. 
Eles lhe informam que trabalham para um ministério da ONU sob 
supervisão de vários países.

Devido a dificuldade para se estabelecer um acordo sobre metas de 
redução de emissção de carbono por parte dos países, este grupo deve 
achar uma outra solução para o problema do aquecimento global. 

Este grupo e formado por cientistas e engenheiros, que desenvolveram
uma solucao (não muito boa mas a melhor possível): Emitir dióxido 
de enxofre na estratosfera, pois este gás reflete grande parte dos 
raios solares e funcionaria como um filtro solar para a Terra, 
invertendo a trajetória de aumento de temperatura. A idéia é construir 
torres com canos enormes que levariam o dióxido de enxofre até a estratosfera. 
Estes são chamados de sistemas de emissão. Dependendo do lugar onde 
este gás é liberado, ele cobrirá diferentes partes do céu, devido as 
correntes de ar existentes. Dado uma posição para instalação do sistema 
de emissão, e dado as correntes de ar existentes, pode-se prever com 
certa precisão quais lugares da Terra será coberto por esta camada 
“protetora”. Para a instalação de um sistema de emissão, há um custo 
monetário e ecológico que varia dependendo do lugar onde este é instalado. 
O objetivo e determinar os lugares de instalação dos sistemas de emissão 
de tal forma que toda a Terra seja coberta pelo dióxido de enxofre, e ao 
mesmo tempo se minimize o custo de instalação dos sistemas de emissão. 
O espaco contínuo a ser coberto (a superfície da Terra) será discretizado, 
de tal forma que temos pontos que devem ser cobertos. 

\section{Algortimo e Adaptações}

O algortimo que iremos utilizar para obter um bom resultado para o problema
descrito é um algoritmo do tipo Ant Colony Optimization chamado Ant System.
Ele foi proposto por Dorigo e é muito utilizado em problemas onde o objetivo
é encontrar o rotas.

Algumas adptações serão feitas para podemos utilizar este algoritmo para
resolvermos o problema. O nosso objetivo é encontrar o menor custo para 
construirmos as torres para cobrir todos os pontos. A principio utilizaremos
o custo de cada torre em substituição a distancias, utilizadas no problemas
onde o algoritmo é aplicado. Porém, somente esta substituição não irá
colaborar muito para encontrarmos um valor ótimo ou bom para a função 
objetivo, pois as distancias entre as torres serão iguais entre si.

Outra adaptação que será utilizada será baseada no fato que as torres podem
cobrir pontos iguais. Então iremos alterar a heuristica utilizada para calcular
as distancias entre as torres utilizando o fator obtido pelos pontos cobertos 
iguais entre as torres. Por exemplo, se a torre S1 cobrir as pontos 1, 2 e 3 e 
e a torre S2 cobrir os pontos 2, 3 e 4 a distancia entre as duas torres
será dada por $((a+b)/2)*(c+1)$ onde $a$ e $b$ são os custos de construção de 
cada torre e $c$ o número de pontos cobertos em comum. Esta medida é tomada
para que não seja utilizada somente o custo como base de escolha e acabe escolhendo
torres que cobrem pontos já cobertos.

\section{Atualização dos Ferormonios}

Uma das fases mais importantes do algoritmo Ant System é a chamada Atualização
de Ferormonios. Enquanto as formigas caminham elas liberar hormonios que funcionam
como uma espécie de guia para outras formigas.

\section{Introdução dos Dados}

O programa criado para encontrar um valor ótimo ou tem como entrada de dados
um arquivo texto, este deve conter na primeira linha o número de pontos que
devem ser cobertos pelas torres.

\section{Resultados}\label{sec:figs}

\section{Conclusão}

\end{document}
