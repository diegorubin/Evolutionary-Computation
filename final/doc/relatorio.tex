\documentclass[12pt]{article}

\usepackage{sbc-template}

\usepackage{graphicx,url}

\usepackage[brazil]{babel}   
\usepackage[utf8]{inputenc}  

     
\sloppy

\title{Resolução de um Problema NP-Completo utilizando ACO}

\author{Diego Rubin\inst{1}}


\address{Departamento de Matemática Aplicada e Computação \\
  - Universidade Estadual Paulista "Julio de Mesquita Filho"
  (UNESP)\\
  Rio Claro -- SP -- Brazil
  \email{rubin.diego@gmail.com}
}

\begin{document} 

\maketitle

\begin{abstract}
\end{abstract}

\begin{resumo} 
\end{resumo}


\section{Introdução}

\section{O Problema} \label{sec:firstpage}

O Problema que iremos abordar é o seguinte:

Você possui uma empresa reconhecida na area de otimizacao e algoritmos. 
Um dia e chamado por telefone para uma conversa urgente com um grupo 
de pessoas que está trabalhando em um grande projeto: “Salvar o mundo”. 
Eles lhe informam que trabalham para um ministério da ONU sob 
supervisão de vários países.

Devido a dificuldade para se estabelecer um acordo sobre metas de 
redução de emissção de carbono por parte dos países, este grupo deve 
achar uma outra solução para o problema do aquecimento global. 

Este grupo e formado por cientistas e engenheiros, que desenvolveram
uma solucao (não muito boa mas a melhor possível): Emitir dióxido 
de enxofre na estratosfera, pois este gás reflete grande parte dos 
raios solares e funcionaria como um filtro solar para a Terra, 
invertendo a trajetória de aumento de temperatura. A idéia é construir 
torres com canos enormes que levariam o dióxido de enxofre até a estratosfera. 
Estes são chamados de sistemas de emissão. Dependendo do lugar onde 
este gás é liberado, ele cobrirá diferentes partes do céu, devido as 
correntes de ar existentes. Dado uma posição para instalação do sistema 
de emissão, e dado as correntes de ar existentes, pode-se prever com 
certa precisão quais lugares da Terra será coberto por esta camada 
“protetora”. Para a instalação de um sistema de emissão, há um custo 
monetário e ecológico que varia dependendo do lugar onde este é instalado. 
O objetivo e determinar os lugares de instalação dos sistemas de emissão 
de tal forma que toda a Terra seja coberta pelo dióxido de enxofre, e ao 
mesmo tempo se minimize o custo de instalação dos sistemas de emissão. 
O espaco contínuo a ser coberto (a superfície da Terra) será discretizado, 
de tal forma que temos pontos que devem ser cobertos (veja Figura). 

\section{Algortimo}


\section{Restrições}


\section{Resultados}\label{sec:figs}

\section{Conclusão}

\end{document}
